%------------------------------
% CHAPTER: Mathematical Foundation
%------------------------------
\chapter{Mathematical Foundation and Notation}
\label{ch:Math}
There are several mathematical concepts central to this research that we will briefly explain here.
The notation used in the rest of this thesis is also explained here for clarity.

\section{Algebraic Structures}
\subsection{Groups}
An algebraic \emph{group} is a set of elements $G$ together with a binary operation $*$ (e.g.\ multiplication or addition) that satisfies the following properties:
\begin{enumerate}
\item \emph{Associativity}. The operation is associative, i.e.\ $(a*b)*c = a*(b*c)$ for all $a, b, c \in G$.
\item \emph{Closure}. $G$ is closed under $*$; that is, $a*b \in G$ for all $a,b \in G$.
\item \emph{Identity}. There is an element $e \in G$ (called the \emph{identity}) such that $a*e = e*a = a$ for all $a \in G$.
\item \emph{Inverses}. For each element $a \in G$ there exists an element $a^{-1} \in G$ (called the \emph{inverse} of $a$) such that $a*a^{-1} = a^{-1}*a = e$.
\end{enumerate}
A group may also have the property that $a*b = b*a$ for all $a, b \in G$.
In other words, it is commutative.
We call these groups \emph{abelian} groups.
The most common example of a group is probably $\mathbb{Z}$, the set of integers under addition, where the identity $e = 0$.

\subsection{Rings}
An algebraic \emph{ring} is a set of elements $R$ together with two binary operations $\cdot$ (called multiplication, with $a \cdot b$ denoted $ab$ for brevity) and $+$ (called addition) that satisfies the following properties:
\begin{enumerate}
\item $R$ is an abelian group under addition; its identity is called $0$.
\item \emph{Associativity}. Multiplication and addition are both associative. 
\item \emph{Distributivity}. $a(b+c) = ab + ac$ and $(b+c)a = ba + ca$; that is, multiplication distributes over addition.
\end{enumerate}
A ring is called abelian if multiplication also commutes over it. 
The most common example of a ring is also $\mathbb{Z}$, but now with multiplication as well as addition.

\subsection{Fields}
An algebraic \emph{field} is a set of elements $\mathbb{F}$ together with the operations $\cdot$ and $+$ such that:
\begin{enumerate}
\item $\mathbb{F}$ is an abelian ring.
\item $\mathbb{F}$ is an abelian group under multiplication; its identity is called $1$.
\end{enumerate}

Groups, rings, and fields (all of which are \emph{algebraic structures}) may be of finite or infinite cardinality.
The cardinality of an algebraic structure $G$ is called its \emph{order} and is denoted $|G|$.

We use an exponential representation as shorthand for applying an operation $*$ to the same element.
For example, $a*a*a = a^3$.
For some structure $G$, the \emph{order} of an element $a \in G$ is the smallest integer $k$ such that $a^k = e$.
For structures of finite order, it is a well-known result (Lagrange's theorem) that the order of an element divides the order of the structure \cite{Gallian2012_AbstractAlgebra}.

\subsection{Galois Fields}

Finite fields are also commonly referred to as Galois fields (GFs).
It is another well-known result that all Galois fields are of prime power order.
A Galois field of order $p^k$ is denoted GF($p^k$) or $\mathbb{F}_{p^k}$, where $p$ is prime.
$p$ is called the \emph{characteristic} of the field and it is the smallest positive integer such that $a^p = 0$.
Fields with $p=2$ are typically of the most interest to cryptographers, and they are often called binary Galois fields.
$k$ is called the \emph{degree} of the field.

When the degree of a field is greater than one, its elements can be represented as polynomials belonging to the set of equivalence classes modulo an irreducible polynomial $f(x)$. 
The degree of this polynomial, denoted $\deg(f(x))$, is equal to the degree of the field.
We denote a GF of order $p^k$ along with its irreducible polynomial as $\mathrm{GF}(p^k) / \left\langle f(x) \right\rangle$.
Elements in the field are represented using a polynomial basis in the indeterminant $x$ with the form
\begin{equation*}
a = \alpha_{k-1}x^{k-1} + \alpha_{k-2}x^{k-2} + \ldots + \alpha_1 x + \alpha_0,
\end{equation*}
where $\alpha_i \in \mathbb{Z}_p$.
For example, in a binary Galois field we have $\alpha_i \in \mathbb{Z}_2$.
Since operations in the GF are done modulo $f(x)$, any element $a \in \mathrm{GF}(p^k)$ must satisfy $\deg(a) < k$.

For a binary Galois field we often represent elements in binary or hex format, as this is how they will be implemented in practice.
For example, consider the element $x^{15} + x^3 + x^2 + 1$ in \gfsixteen.
This element can be represented in hex as $\mathrm{0x800d}$, where the most-significant bit (MSB) corresponds to $x^{15}$. 

Addition in finite fields is done by performing element-wise addition over $\mathbb{Z}_p$.
In a binary Galois field, this is equivalent to the bitwise XOR operation.
Multiplication is performed by doing polynomial multiplication as usual and then reducing modulo $f(x)$ if needed.
There are methods of optimizing this operation in both software and hardware.

Interpreting binary strings as elements of a binary Galois field is only useful for certain operations. Other times, we interpret portions of our state as arbitrary bitstrings for which certain other operations are defined.
 
\section{Bitstrings}
Bitstrings are strings of elements in $\Ztwo$; in other words, they are binary strings.
We say $s \in \Ztwo^n$ if $s$ is a bitstring of length $n$.
$\Ztwo^*$ indicates bitstrings of arbitrary but finite length and $\Ztwo^{\infty}$ denotes bitstrings of infinite length.

\subsection{Operations}
There are several relevant operations we define for bitstrings:
\begin{itemize}
\item $\lfloor s \rfloor_{\ell}$ indicates the truncation of a bitstring $s$ to $\ell$ bits
\item $s||t$ indicates the concatenation of bitstrings $s$ and $t$, with the resulting length being $|s| + |t|$
\item $s \lll r$ indicates the logical left rotation of bitstring $s$ by $r$ bits 
\item $s \ggg r$ indicates the logical right rotation of bitstring $s$ by $r$ bits 
\item $s \oplus t$ indicates the exclusive-or (XOR, or bitwise addition modulo $2$) of bitstrings $s$ and $t$, where $|s| = |t|$ is assumed
\item $\mathbf{HW}(s)$ indicates the Hamming weight of bitstring $s$
\item $\mathbf{HD}(s,t)$ indicates the Hamming distance between bitstrings $s$ and $t$, where $|s| = |t|$ is assumed
\end{itemize}

The \emph{Hamming weight} of a string is defined as the number of nonzero elements it contains.
The \emph{Hamming distance} between two strings is the number of elements that differ between them.
Note that for elements of $\mathbb{Z}_2^n$, 
\begin{equation*}
\mathbf{HD}(s,t) = \mathbf{HW}(s \oplus t).
\end{equation*}

For this work, a bitstring of length $16$ is called a \emph{word}.
The state of our cryptosystem, as we will see, is an element of $\mathbb{Z}_2^{512}$ that may be considered at times as the concatenation of $32$ contiguous words.
Where it matters, the MSB will be specified for a given operation.

\section{Shannon's Information Theory}
It is widely recognized that Claude Shannon laid the foundation for modern symmetric key cryptography in the late 1940s through two seminal papers on information theory and secrecy systems.
His major relevant contributions include the notions of entropy, confusion, and diffusion \cite{Shannon1948_AMTOC}\cite{Shannon1949_CTOSS}.

\subsection{Entropy}
\emph{Entropy}, as defined by Shannon, is the measure of the amount of information in some message $M$. It is denoted as $H(M)$ and is quantitatively defined as  
\begin{equation*}
H(M) = \log_2 n,
\end{equation*}
where $n$ is the number of possible \emph{meanings} of the message.
For example, the entropy of a message containing only information about what month it is is $\log_2(12) \approx 3.58$ since there are $12$ months.
Another way to look at this is that exactly $\lceil 3.58 \rceil = 4$ bits are required to minimally encode in binary what month it is.

Entropy also measures \emph{uncertainty}.
The uncertainty of a message $M$ is the number of plaintext bits needed to be recovered from ciphertext in order to learn the plaintext.
For example, if we know that the plaintext is either ``HEADS'' or ``TAILS'' when provided with the ciphertext ``ASDFQJWE'', then the uncertainty is unity.
We need only learn one bit (if chosen correctly) to learn the plaintext.

We often discuss the entropy of cryptosystems with respect to their key space.
For example, a symmetric key system with a $128$-bit key should have an entropy of $128$ bits.

\subsection{Transformations}
A \emph{transformation} is simply a function
\begin{equation*}
t \from X \to Y,
\end{equation*} 
where $X$ is the domain and $Y$ is the codomain.
Some transformations on a space are \emph{entropy-preserving}; that is, if an input $x$ has entropy $H(x)$ before the transformation, it has the exact same entropy (and thus uncertainty) after the transformation.
An obvious requirement for entropy-preserving transformations is that they are bijective.
We call a bijective transformation in which the domain is equal to the codomain a \emph{permutation}. 
This notion is of central importance to the remainder of this thesis.

\subsection{Confusion and Diffusion}
Also of central importance to this thesis are Shannon's qualitative notions of confusion and diffusion.
\emph{Confusion} is a generic method used to obscure the relationship between the plaintext and ciphertext.
For example, substitution provides confusion.
\emph{Diffusion} is a generic method that aims to dissipate the redundancy of plaintext throughout the ciphertext.
For example, bitwise permutations provide diffusion.
In theory, extremely large-scale substitutions are sufficient to create a secure symmetric key system.
However, in practical block cipher systems both confusion and diffusion are required for efficiency reasons \cite{Schneier1996_Applied}.

