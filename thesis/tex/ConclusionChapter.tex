%------------------------------
% CHAPTER:Conclusions
%------------------------------
\chapter{Conclusions and Future Work}
\label{ch:Conclusions}

\section{Conclusions}
In this thesis we designed an authenticated encryption algorithm based on the duplex construction that we believe to be highly secure.
This algorithm can be customized on a per-user or per-application basis by following the guidelines we presented as part of its specification.
We believe that this is the only customizable authenticated encryption algorithm to date.
Furthermore, we believe that this is the first cryptosystem to use a $16$-bit S-box.
We have shown that this large S-box, when combined with good diffusion, provides superior resistance against differential and linear cryptanalysis due to its small differential probabilities and linear biases.
Because this S-box is based on bijective power mappings over a Galois field, it can be efficiently implemented in hardware (like the rest of our algorithm) and does not require a look-up table.

Following the algorithm specification, we presented a survey of state of the art cryptanalysis techniques and assessed our algorithm's resistance against them.
Additionally, we provided a comprehensive statistical evaluation of the cryptosystem through the use of existing tools as well as our own.
Our algorithm demonstrates no significant non-random statistical behavior and its underlying permutation achieves full diffusion on the entire $512$-bit state after only three rounds.

\section{Future Work}
The scope of this thesis is quite wide.
As a result, there are many items that we would love to research further but have so far been unable to.
Several of these items are provided here.

\begin{enumerate}
\item \textbf{Cryptanalysis:}
The duplex construction we use guarantees security against generic attacks, and our underlying permutation is designed for resistance against many powerful cryptanalysis techniques.
However, as with all cryptosystems, further cryptanalysis is always welcome and encouraged.
In particular, it is likely possible to prove better bounds for our resistance against differential and linear cryptanalysis.
So far, we have proved only the absolute worst case.
Furthermore, it could be interesting to dig deeper into the fairly new bit-oriented integral attacks.
While we strongly believe our permutation to be secure against such an attack, it would be a worthwhile exercise to attempt to prove this resistance.

\item \textbf{Hardware Design:} 
Due to time constraints we have only provided a software model of the algorithm in this thesis.
In practice, our algorithm is intended for hardware (e.g.\ FPGA) implementation.
One of the natural next steps is to implement the construction in hardware and benchmark its performance.

\item \textbf{S-boxes:} 
The particular instantiation of our algorithm described in Chapter~\ref{ch:AlgorithmSpec} uses a specific S-box based on bijective power mappings. 
We analyzed this S-box to determine its maximum differential probability and maximum linear bias.
There are several other similar S-boxes provided in \cite{Wood2013_SboxThesis}.
It would be helpful to also analyze these other S-boxes in order to determine if their statistical properties are much different.
If the maximum probabilities and biases are very close to those of our particular S-box, these other S-boxes could be swapped in as a customization without the need for determining new security margins (and possibly adjusting the number of rounds).
In addition, it would be extremely fruitful to find completely new cryptographically secure $16$-bit S-boxes that are efficiently implementable in hardware.

\item \textbf{Large Mixers:}
Part of the reason that we use a $2 \times 2$ matrix multiplication for our mixer is that it is extremely efficient to implement in hardware.
It is well known that the maximum branch number of an $m \times m$ matrix is $m+1$. 
We achieve the maximum differential and linear branch number of three with this mixer.
Larger matrices are capable of achieving higher branch numbers, and so it would be interesting to further explore them.
In particular, Maximum Distance Separable (MDS) matrices such as the one used in AES are capable of achieving the maximum branch number.
However, the construction of large MDS matrices that are efficiently implementable is a difficult problem \cite{Daemen2001_WideTrail}.
An in-depth analysis of the tradeoff between a large MDS matrix with fewer rounds and a small matrix with more rounds is welcome for future work.

\item \textbf{ARX-Based Mixers:}
As part of the work that did not make it into the final AE algorithm presented here, we analyzed many ARX-based mixers.
None of these mixers were able to achieve a branch number greater than two (the worst possible).
So far it is unclear what an ARX-based mixer with branch number equal to three would look like, if it even exists.
While we believe that our matrix multiplication-based mixer is an excellent choice, it would still be interesting to further analyze ARX-based mixers in an attempt to generalize their characteristics with respect to differential and linear cryptanalysis.

\end{enumerate}

