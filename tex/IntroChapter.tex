%------------------------------
% CHAPTER:Introduction
%------------------------------
\chapter{Introduction}
\label{ch:Intro}
The overarching goal of cryptography is to enable people to communicate privately over an insecure channel in the presence of adversaries.
This notion may be abstracted further; for example, the sending and receiving party may be the same person writing to and reading from a storage medium over time.
Two requirements for achieving this goal are encryption and authentication.
Encryption provides \textit{confidentiality} while authentication provides data \textit{integrity} and assurance of message origin.

Encryption without authentication has been shown to be insecure in practical instances several times in recent history.
For example, Wireless Equivalence Privacy (WEP), an algorithm for ``securing'' communications over wireless networks, is badly broken due (in part) to a lack of proper authentication \cite{Borisov2001_WEP}.
In fact, it is so badly broken that people with no cryptographic expertise can use existing tools on standard consumer machines to break into networks that use $104$-bit WEP in less than $60$ seconds \cite{Tews2007_BreakingWEP}.

Another example of a flaw in prevalent systems due to a lack of proper authentication came in 2002.
An attack was found on the CBC mode of encryption employed by protocols such as SSL and IPSEC.
The attack could have been avoided had an authenticated encryption scheme been used \cite{Vaudenay2002_CBC_Flaws}.

%------------------------------
% SECTION: Motivation
%------------------------------
\section{Motivation}
Many authenticated encryption algorithms are in existence today, but they are often unsatisfactory in terms of performance, security, or ease of use.
Some algorithms require two passes per block of plaintext to encrypt and authenticate.
This is generally undesirable because it often means a much slower algorithm.
Other algorithms have been shown to be insecure or difficult to use properly.
Many algorithms, such as the ones based on generic composition, require two keys.
This should be avoided when possible because key management is a difficult problem.

Furthermore, a new authenticated encryption algorithm is required that meets the stringent requirements of government and military applications.
Such algorithms are not typically in the public domain.
The goal of this is partially to reduce or eliminate academic interest in cryptanalyzing the algorithm and publishing results.
This stance is highly controversial.
Still, the security of such algorithms depends entirely on the secrecy of the key and not on the secrecy of the algorithm.
The assumption is still made that the enemy knows the details of the algorithm being used at any time \cite{Kurdziel2002_BaselineRequirements}.

For this reason, there is a need for a customizable authenticated encryption algorithm.
This algorithm should remain secure as long as customizations are made within certain guidelines.
The result is an algorithm which can be made unique on a per-user or per-application basis without the effort of cryptanalyzing every specific instantiation.
We present such an algorithm here.
First, it is important to understand how authenticated encryption has been evolving over the years.

%------------------------------
% SECTION: Previous Efforts
%------------------------------
\section{Overview of Authenticated Encryption}
\subsection{AE Through Generic Composition}
The na{\"i}ve approach to authenticated encryption is generic composition.
In this construction, one simply computes the ciphertext under a given key and computes the MAC under a different key.
These operations can be done in any order or at the same time.
There are thus three generic composition types: \emph{Encrypt-and-MAC}, \emph{MAC-then-Encrypt}, and \emph{Encrypt-then-MAC}.
The names here are straightforward and describe exactly the order of operations.
In Encrypt-and-MAC, the encryption and authentication operations are performed in parallel and the ciphertext and MAC are concatenated together at the end.
MAC-then-Encrypt computes the MAC, appends it to the plaintext, and then encrypts the result.
Encrypt-then-MAC computes the ciphertext first and then computes the MAC over the ciphertext. 

There is much ongoing debate about which generic composition method is best and each method has pros and cons in terms of security and performance.
As a simple example, consider the performance of each method at a high level of abstraction.
In Encrypt-and-MAC, the encryption (decryption) and authentication can happen in parallel.
If using MAC-then-Encrypt, decryption must be performed before the MAC can be verified.
The MAC can be checked before decryption in the case of Encrypt-then-MAC, thus saving the performance hit of decryption for messages in which authentication fails \cite{Ferguson2010_CryptographyEngineering}.

\subsection{AE Modes of Operation}
There are many existing block cipher modes of operation that provide authenticated encryption.
The main difference between a generic composition and a mode of operation for authenticated encryption is that a generic composition always requires two keys, while a mode of operation generally requires only one.
This is a huge benefit because key management is a difficult task, so it is desirable to limit the number of keys required for a cryptographic algorithm.

\subsubsection{Patent-Encumbered Modes}
The first notable modes of operation for AE were created by Jutla, an IBM researcher, in the year 2000.
The two modes are called Integrity Aware Cipher Block Chaining (IACBC) and Integrity Aware Parallelizable Mode (IAPM) \cite{Jutla2001_AE}.
IACBC, as the name suggests, was inspired by the CBC mode of operation.
This mode is highly serial in nature.
IAPM was inspired by the Electronic Codebook (ECB) mode, which is highly parallelizable but insecure when used directly.
Jutla's modes are historically noteworthy in that they provide authenticated encryption in a single pass with minimal overhead; however, they come with many disadvantages.
For example, they do not support additional authenticated data (e.g.\ headers that are only authenticated, not encrypted).
An even bigger problem is that they are highly patent encumbered.
On top of all this, they require two keys and the underlying block cipher decryption mode.
As a result of the lack of popularity, there has not been much cryptanalysis performed on IACBC or IAPM.

In 2001, Rogaway et al.\ \cite{Rogaway2003_OCB} introduced a new AE mode called Offset Codebook (OCB) mode.
This mode is based on IAPM, but with many added features.
OCB requires only a single key and provides support for additional authenticated data.
Furthermore, it is highly parallelizable and also a single-pass mode.
The main disadvantages of OCB are that it also requires the block cipher decryption mode and that it is also patent encumbered. 

\subsubsection{Patent-Free Modes}
Counter with CBC-MAC (CCM) mode was created by Whiting et al.\ in 2003 as a response to OCB's patented status \cite{Whiting2003_CCM}. CCM mode is patent-free but has many disadvantages.
Like all of the patent-free modes that will be discussed here, it requires two passes under a single key.
It does not, however, require the decryption mode of the underlying block cipher since it is built on Counter mode.
CCM mode does not support block sizes other than $128$ bits and does not support stream processing.
This is a significant disadvantage because stream processing is often exactly what a AE construction is used for (e.g.\ streaming network traffic).
Rogaway and Wagner present a detailed analysis of all of the disadvantages of CCM mode in \cite{Rogaway2003_CritiqueOfCCM}.

Carter-Wegman + Counter (CWC) mode was introduced by Kohno et al.\ in 2004 as an attempt to improve upon the deficiencies of CCM mode \cite{Kohno2004_CWC}.
Like CCM mode, it is a single-key mode that requires two passes.
Unlike CCM, the second pass does not consist of any invocations of the underlying block cipher.
It instead uses what is called a Carter-Wegman (CW) universal hash function.
This CW hash function operates over a prime field and thus requires integer multipliers which consume a large amount of area when implemented in hardware.

McGrew and Viega created Galois/Counter Mode (GCM) in 2004 in an effort to improve upon the efficiency of CWC mode \cite{McGrew2004_GCM}.
Instead of operating in a prime field, GCM performs multiplication in a binary Galois field.
As a result, it consumes much less area in hardware.
Like all other patent-free modes described here, it is single-key and two-pass.
It should be noted that some people may argue that GCM is single-pass because only the first pass calls the underlying block cipher.
GCM has seen widespread adoption by the cryptographic community.
It is arguably the most used mode of operation for AE in existence today.
Indeed, GCM is a main reason for the advent of Intel's carry-less multiplication instruction (PCLMULQDQ) in their 2010 line of processors \cite{Gueron2010_IntelGCM}.

EAX mode was introduced in 2004 by Bellare et al.\ as another alternative to CCM mode \cite{Bellare2004_EAX}.
Like CCM mode, it is not parallelizable.
Like all other modes listed here, it is single-key and two-pass.
EAX mode is derived from a generic composition called EAX2 by the authors.
EAX2 mode leaves the Pseudorandom Function (PRF) used for MAC computation and the underlying block cipher mode used for encryption unspecified.
EAX2 uses two keys.
EAX is a specific instantiation of EAX2 that uses only a single key.
Like GCM, EAX mode has also seen widespread adoption.

\subsection{Other Existing AE Constructions}
\subsubsection{Stream Cipher Based}
There are two notable stream ciphers that provide authenticated encryption: Helix \cite{Whiting2005_Helix} and Phelix \cite{Whiting2005_Phelix}, both from Whiting et al.\ Phelix is the successor of Helix and was submitted to the recent eSTREAM competition which was a competition aimed at yielding better stream ciphers.
Phelix, though very fast, is not currently a viable AE candidate because of its vulnerability to differential-linear attacks as exposed by Wu and Preneel \cite{Wu2007_PhelixAttack}.
The security flaw relates to lack of resistance to misuse of the stream cipher.
Still, new stream cipher based AE constructions are probable in the future.

\subsubsection{Duplex Constructions}
The duplex construction is based on a sponge construction and has promising applications to authenticated encryption.
This construction has many extremely desirable features.
It is single-key, single-pass, and supports additional authenticated data and intermediate MACs.
For these reasons, the duplex construction forms a foundation for this thesis; see Chapter~\ref{ch:SpongeAndDuplex} for an in-depth treatment.

At the time of writing there is a new AE competition going on called CAESAR (Competition for Authenticated Encryption: Security, Applicability, and Robustness). 
Of the $57$ first-round candidates, there are nine duplex-based submissions that are not withdrawn:
Artemia, Ascon, ICEPOLE, Ketje, Keyak, NORX, PRIMATEs, STRIBOB, and $\pi$-Cipher \cite{Bernstein2014_CAESAR_Submissions}.
It is far too early to tell which of these, if any, will be found to be secure and efficient enough for practical use.
Furthermore, none of these submissions are readily customizable at an algorithmic level.
Therefore they are not suitable for our purposes.

%------------------------------
% SECTION: Contribution
%------------------------------
\section{Our Contributions}
The main contribution of this thesis is a novel, customizable authenticated encryption algorithm based on the duplex construction.
This construction is suitable for hardware (e.g.\ FPGA) implementation.
An itemized list of specific contributions is listed here:
\begin{enumerate}
\item Authenticated encryption algorithm specification
  \begin{enumerate}
    \item Includes customization suggestions and guidelines
  \end{enumerate}
\item Software model of the algorithm (C/C++)
  \begin{enumerate}
    \item Includes various tests and test data generators
    \item Includes \gfsixteen library for algorithm verification
  \end{enumerate}
\item Tool for finding suitable bitwise permutations for algorithm customization (Python)
\item Survey of relevant cryptanalysis techniques and cryptanalysis results
\item Statistical test results 
\item Tool for ensuring acceptable distribution of \pvals (Python)
\end{enumerate}

The rest of this thesis is organized as follows.
In Chapter~\ref{ch:Math} we discuss some mathematical concepts that are central to this thesis.
In addition, we explain the notation used for the remainder of this document.
Chapter~\ref{ch:SpongeAndDuplex} explains in detail the sponge and duplex constructions as well as their resistance to generic attacks. 
The duplex construction forms the basis for the algorithm designed in this thesis.
In Chapter~\ref{ch:AlgorithmSpec} we provide a full specification for the AE algorithm as well as providing guidelines for customizations that can be made without reducing our security margin.
Cryptanalysis of our construction is considered in Chapter~\ref{ch:Cryptanalysis}, where we provide a survey of potentially relevant attacks.
A variety of statistical tests and their results for our algorithm are presented in Chapter~\ref{ch:StatisticalTesting}.
Finally, we conclude in Chapter~\ref{ch:Conclusions} and provide some ideas for future work.

