%------------------------------
% SECTION: Introduction
%------------------------------
\section{Introduction}
The overarching goal of symmetric key cryptography is to enable people to communicate privately over an insecure channel in the presence of adversaries.
Two fundamental requirements for achieving this goal are encryption and authentication.
Encryption provides \emph{confidentiality} while authentication provides data \emph{integrity} and assurance of message origin.

Many authenticated encryption algorithms are in existence today, but they are often unsatisfactory in terms of performance, security, or ease of use.
Some algorithms require two passes per block of plaintext to encrypt and authenticate.
This is generally undesirable because it often means a much slower algorithm.
Other algorithms have been shown to be insecure or difficult to use properly.
Many algorithms, such as the ones based on generic composition, require two keys.
This should be avoided when possible because key management is a difficult problem.

Furthermore, a new authenticated encryption algorithm is needed that meets the stringent requirements of government and military applications.
Such algorithms are not typically in the public domain.
The goal of this is partially to reduce or eliminate academic interest in cryptanalyzing the algorithm and publishing results.
This stance is highly controversial.
Still, the security of such algorithms depends entirely on the secrecy of the key and not on the secrecy of the algorithm.
The assumption is still made that the enemy knows the details of the algorithm being used at any time \cite{Kurdziel2002_BaselineRequirements}.

For this reason, there is a need for a customizable authenticated encryption algorithm.
This algorithm should remain secure as long as customizations are made within certain guidelines.
The result is an algorithm which can be made unique on a per-user or per-application basis without the effort of cryptanalyzing every specific instantiation.
We present such an algorithm here.

